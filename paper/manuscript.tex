% Options for packages loaded elsewhere
\PassOptionsToPackage{unicode}{hyperref}
\PassOptionsToPackage{hyphens}{url}
%
\documentclass[
  english,
  man,floatsintext]{apa6}
\usepackage{lmodern}
\usepackage{amssymb,amsmath}
\usepackage{ifxetex,ifluatex}
\ifnum 0\ifxetex 1\fi\ifluatex 1\fi=0 % if pdftex
  \usepackage[T1]{fontenc}
  \usepackage[utf8]{inputenc}
  \usepackage{textcomp} % provide euro and other symbols
\else % if luatex or xetex
  \usepackage{unicode-math}
  \defaultfontfeatures{Scale=MatchLowercase}
  \defaultfontfeatures[\rmfamily]{Ligatures=TeX,Scale=1}
\fi
% Use upquote if available, for straight quotes in verbatim environments
\IfFileExists{upquote.sty}{\usepackage{upquote}}{}
\IfFileExists{microtype.sty}{% use microtype if available
  \usepackage[]{microtype}
  \UseMicrotypeSet[protrusion]{basicmath} % disable protrusion for tt fonts
}{}
\makeatletter
\@ifundefined{KOMAClassName}{% if non-KOMA class
  \IfFileExists{parskip.sty}{%
    \usepackage{parskip}
  }{% else
    \setlength{\parindent}{0pt}
    \setlength{\parskip}{6pt plus 2pt minus 1pt}}
}{% if KOMA class
  \KOMAoptions{parskip=half}}
\makeatother
\usepackage{xcolor}
\IfFileExists{xurl.sty}{\usepackage{xurl}}{} % add URL line breaks if available
\IfFileExists{bookmark.sty}{\usepackage{bookmark}}{\usepackage{hyperref}}
\hypersetup{
  pdftitle={The Sound of Teaching Music: Expert Pianists' Performance Modulations for Novices},
  pdflang={en-EN},
  pdfkeywords={teaching, expertise, skill transmission, artistic expression, music},
  hidelinks,
  pdfcreator={LaTeX via pandoc}}
\urlstyle{same} % disable monospaced font for URLs
\usepackage{graphicx}
\makeatletter
\def\maxwidth{\ifdim\Gin@nat@width>\linewidth\linewidth\else\Gin@nat@width\fi}
\def\maxheight{\ifdim\Gin@nat@height>\textheight\textheight\else\Gin@nat@height\fi}
\makeatother
% Scale images if necessary, so that they will not overflow the page
% margins by default, and it is still possible to overwrite the defaults
% using explicit options in \includegraphics[width, height, ...]{}
\setkeys{Gin}{width=\maxwidth,height=\maxheight,keepaspectratio}
% Set default figure placement to htbp
\makeatletter
\def\fps@figure{htbp}
\makeatother
\setlength{\emergencystretch}{3em} % prevent overfull lines
\providecommand{\tightlist}{%
  \setlength{\itemsep}{0pt}\setlength{\parskip}{0pt}}
\setcounter{secnumdepth}{-\maxdimen} % remove section numbering
% Make \paragraph and \subparagraph free-standing
\ifx\paragraph\undefined\else
  \let\oldparagraph\paragraph
  \renewcommand{\paragraph}[1]{\oldparagraph{#1}\mbox{}}
\fi
\ifx\subparagraph\undefined\else
  \let\oldsubparagraph\subparagraph
  \renewcommand{\subparagraph}[1]{\oldsubparagraph{#1}\mbox{}}
\fi
% Manuscript styling
\usepackage{upgreek}
\captionsetup{font=singlespacing,justification=justified}

% Table formatting
\usepackage{longtable}
\usepackage{lscape}
% \usepackage[counterclockwise]{rotating}   % Landscape page setup for large tables
\usepackage{multirow}		% Table styling
\usepackage{tabularx}		% Control Column width
\usepackage[flushleft]{threeparttable}	% Allows for three part tables with a specified notes section
\usepackage{threeparttablex}            % Lets threeparttable work with longtable

% Create new environments so endfloat can handle them
% \newenvironment{ltable}
%   {\begin{landscape}\begin{center}\begin{threeparttable}}
%   {\end{threeparttable}\end{center}\end{landscape}}
\newenvironment{lltable}{\begin{landscape}\begin{center}\begin{ThreePartTable}}{\end{ThreePartTable}\end{center}\end{landscape}}

% Enables adjusting longtable caption width to table width
% Solution found at http://golatex.de/longtable-mit-caption-so-breit-wie-die-tabelle-t15767.html
\makeatletter
\newcommand\LastLTentrywidth{1em}
\newlength\longtablewidth
\setlength{\longtablewidth}{1in}
\newcommand{\getlongtablewidth}{\begingroup \ifcsname LT@\roman{LT@tables}\endcsname \global\longtablewidth=0pt \renewcommand{\LT@entry}[2]{\global\advance\longtablewidth by ##2\relax\gdef\LastLTentrywidth{##2}}\@nameuse{LT@\roman{LT@tables}} \fi \endgroup}

% \setlength{\parindent}{0.5in}
% \setlength{\parskip}{0pt plus 0pt minus 0pt}

% \usepackage{etoolbox}
\makeatletter
\patchcmd{\HyOrg@maketitle}
  {\section{\normalfont\normalsize\abstractname}}
  {\section*{\normalfont\normalsize\abstractname}}
  {}{\typeout{Failed to patch abstract.}}
\makeatother
\shorttitle{The Sound of Teaching Music}
\author{Atsuko Tominaga\textsuperscript{1}, Günther Knoblich\textsuperscript{1}, \& Natalie Sebanz\textsuperscript{1}}
\affiliation{
\vspace{0.5cm}
\textsuperscript{1} Department of Cognitive Science, Central European University}
\authornote{

Correspondence concerning this article should be addressed to Atsuko Tominaga, Quellenstraße 51, 1100 Vienna, Austria. E-mail: Tominaga\_Atsuko@phd.ceu.edu}
\keywords{teaching, expertise, skill transmission, artistic expression, music\newline\indent Word count: X}
\usepackage{lineno}

\linenumbers
\usepackage{csquotes}
\ifxetex
  % Load polyglossia as late as possible: uses bidi with RTL langages (e.g. Hebrew, Arabic)
  \usepackage{polyglossia}
  \setmainlanguage[]{english}
\else
  \usepackage[shorthands=off,main=english]{babel}
\fi
\ifluatex
  \usepackage{selnolig}  % disable illegal ligatures
\fi
\newlength{\cslhangindent}
\setlength{\cslhangindent}{1.5em}
\newenvironment{cslreferences}%
  {}%
  {\par}

\title{The Sound of Teaching Music: Expert Pianists' Performance Modulations for Novices}

\date{}

\abstract{
One or two sentences providing a \textbf{basic introduction} to the field, comprehensible to a scientist in any discipline.

Two to three sentences of \textbf{more detailed background}, comprehensible to scientists in related disciplines.

One sentence clearly stating the \textbf{general problem} being addressed by this particular study.

One sentence summarizing the main result (with the words ``\textbf{here we show}'' or their equivalent).

Two or three sentences explaining what the \textbf{main result} reveals in direct comparison to what was thought to be the case previously, or how the main result adds to previous knowledge.

One or two sentences to put the results into a more \textbf{general context}.

Two or three sentences to provide a \textbf{broader perspective}, readily comprehensible to a scientist in any discipline.
}

\begin{document}
\maketitle

\hypertarget{introduction}{%
\section{1. Introduction}\label{introduction}}

\hypertarget{experiment-1}{%
\section{2. Experiment 1}\label{experiment-1}}

\hypertarget{method}{%
\subsection{2.1. Method}\label{method}}

\hypertarget{participants}{%
\subsubsection{2.1.1. Participants}\label{participants}}

\hypertarget{apparatus-and-stimuli}{%
\subsubsection{2.1.2. Apparatus and stimuli}\label{apparatus-and-stimuli}}

\hypertarget{procedure}{%
\subsubsection{2.1.3. Procedure}\label{procedure}}

\hypertarget{design-and-data-analysis}{%
\subsubsection{2.1.4. Design and data analysis}\label{design-and-data-analysis}}

\hypertarget{results-articulation}{%
\subsection{2.2. Results (Articulation)}\label{results-articulation}}

\hypertarget{interonset-intervals-iois}{%
\subsubsection{2.2.1. Interonset Intervals (IOIs)}\label{interonset-intervals-iois}}

\hypertarget{key-overlap-time-kot}{%
\subsubsection{2.2.2. Key-Overlap Time (KOT)}\label{key-overlap-time-kot}}

\hypertarget{key-velocity-kv}{%
\subsubsection{2.2.3. Key Velocity (KV)}\label{key-velocity-kv}}

\hypertarget{kv-transition-points}{%
\subsubsection{2.2.4. KV Transition Points}\label{kv-transition-points}}

\hypertarget{results-dynamics}{%
\subsection{2.3. Results (Dynamics)}\label{results-dynamics}}

\hypertarget{interonset-intervals-iois-1}{%
\subsubsection{2.3.1. Interonset Intervals (IOIs)}\label{interonset-intervals-iois-1}}

\hypertarget{key-velocity-kv-1}{%
\subsubsection{2.3.2. Key Velocity (KV)}\label{key-velocity-kv-1}}

\hypertarget{kv-transition-points-1}{%
\subsubsection{2.3.3. KV Transition Points}\label{kv-transition-points-1}}

\hypertarget{key-overlap-time-kot-1}{%
\subsubsection{2.3.4. Key-Overlap Time (KOT)}\label{key-overlap-time-kot-1}}

\hypertarget{discussion}{%
\subsection{2.4. Discussion}\label{discussion}}

\newpage

\hypertarget{experiment-2}{%
\section{3. Experiment 2}\label{experiment-2}}

\hypertarget{method-1}{%
\subsection{3.1. Method}\label{method-1}}

\hypertarget{participants-1}{%
\subsubsection{3.1.1. Participants}\label{participants-1}}

\hypertarget{apparatus-and-stimuli-1}{%
\subsubsection{3.1.2. Apparatus and stimuli}\label{apparatus-and-stimuli-1}}

\hypertarget{procedure-1}{%
\subsubsection{3.1.3. Procedure}\label{procedure-1}}

\hypertarget{design-and-data-analysis-1}{%
\subsubsection{3.1.4. Design and data analysis}\label{design-and-data-analysis-1}}

\hypertarget{results-articulation-1}{%
\subsection{3.2. Results (Articulation)}\label{results-articulation-1}}

\hypertarget{interonset-intervals-iois-2}{%
\subsubsection{3.2.1. Interonset Intervals (IOIs)}\label{interonset-intervals-iois-2}}

\hypertarget{key-overlap-time-kot-2}{%
\subsubsection{3.2.2. Key-Overlap Time (KOT)}\label{key-overlap-time-kot-2}}

\hypertarget{key-velocity-kv-2}{%
\subsubsection{3.2.3. Key Velocity (KV)}\label{key-velocity-kv-2}}

\hypertarget{kv-transition-points-2}{%
\subsubsection{3.2.4. KV Transition Points}\label{kv-transition-points-2}}

\hypertarget{results-dynamics-1}{%
\subsection{3.3. Results (Dynamics)}\label{results-dynamics-1}}

\hypertarget{interonset-intervals-iois-3}{%
\subsubsection{3.3.1. Interonset Intervals (IOIs)}\label{interonset-intervals-iois-3}}

\hypertarget{key-velocity-kv-3}{%
\subsubsection{3.3.2. Key Velocity (KV)}\label{key-velocity-kv-3}}

\hypertarget{kv-transition-points-3}{%
\subsubsection{3.3.3. KV Transition Points}\label{kv-transition-points-3}}

\hypertarget{key-overlap-time-kot-3}{%
\subsubsection{3.3.4. Key-Overlap Time (KOT)}\label{key-overlap-time-kot-3}}

\hypertarget{discussion-1}{%
\subsection{3.4. Discussion}\label{discussion-1}}

\hypertarget{general-discussion}{%
\section{4. General Discussion}\label{general-discussion}}

\hypertarget{acknowledgements}{%
\section{Acknowledgements}\label{acknowledgements}}

We thank Dávid Csűrös for his help with data collection. This research was supported by the European Research Council under the European Union's Seventh Framework Program (FP7/2007-2013)/ERC (European Research Council) grant agreement no. 609819, SOMICS, and by ERC grant agreement no. 616072, JAXPERTISE.

\hypertarget{declarations-of-interest}{%
\section{Declarations of interest}\label{declarations-of-interest}}

All authors have no conflict of interest.

\newpage

\hypertarget{references}{%
\section{References}\label{references}}

\begingroup
\setlength{\parindent}{-0.5in}
\setlength{\leftskip}{0.5in}

\hypertarget{refs}{}
\begin{cslreferences}
\end{cslreferences}

\endgroup
\raggedbottom

\end{document}
